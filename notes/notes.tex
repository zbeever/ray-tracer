\documentclass{article}

\usepackage[margin=1in]{geometry}
\usepackage{scrextend}
\usepackage{tikz, tkz-euclide}
\usepackage{amsmath}
\usepackage{amssymb}
\usepackage{commath}
\usepackage{sectsty}
\usepackage{array}
\usepackage{etoolbox}
\usepackage{color}
\usepackage{subfiles}
\usepackage{wasysym}

\begin{document}
\section*{Sphere Intersection}
We know that a sphere of radius $r$ is the set of all points a distance $r$ from sphere's center. In vector notation, this is
\[
	(\mathbf{P} - \mathbf{O}_S)\cdot(\mathbf{P} - \mathbf{O}_S) = r^2
\]
where $\mathbf{P}$ is a vector touching a point on the sphere and $\mathbf{O}_S$ is the vector describing the sphere's center.

We describe a ray via
\[
	\mathbf{P}(t) = \mathbf{O}_R + \mathbf{D}t
\]
where $t \in \mathbb{R}$. Substituting this into the implicit equation for our sphere yields
\begin{align*}
		r^2 &= (\mathbf{P}(t) - \mathbf{O}_S)\cdot(\mathbf{P}(t) - \mathbf{O}_S), \\
		&= (\mathbf{O}_R + \mathbf{D}t - \mathbf{O}_S)\cdot(\mathbf{O}_R + \mathbf{D}t - \mathbf{O}_S), \\
		&= \big(\mathbf{D}t + (\mathbf{O}_R - \mathbf{O}_S)\big) \cdot \big(\mathbf{D}t + (\mathbf{O}_R - \mathbf{O}_S)\big), \\
		&= (\mathbf{D}\cdot\mathbf{D})t^2 + 2\mathbf{D}\cdot(\mathbf{O}_R - \mathbf{O}_S)t + (\mathbf{O}_R - \mathbf{O}_S)\cdot(\mathbf{O}_R - \mathbf{O}_S).
\end{align*}
We can define $\mathbf{U} = \mathbf{O}_R - \mathbf{O}_S$, collect everything on one side, and group terms by orders of $t$ to arrive at
\[
	(\mathbf{D}\cdot\mathbf{D})t^2 + (2\mathbf{D}\cdot\mathbf{U})t + \big(\mathbf{U}\cdot\mathbf{U}- r^2\big) = 0
\]
which is a simple quadratic equation in $t$. Its solution is
\[
	t = \frac{-2\mathbf{D}\cdot\mathbf{U} \pm \sqrt{\big(2\mathbf{D}\cdot\mathbf{U}\big)^2 - 4(\mathbf{D}\cdot\mathbf{D})(\mathbf{U}\cdot\mathbf{U} - r^2)}}{2\mathbf{D}\cdot\mathbf{D}}.
\]
or, removing a common factor of $2$, 
\[
t = \frac{-\mathbf{D}\cdot\mathbf{U} \pm \sqrt{\big(\mathbf{D}\cdot\mathbf{U}\big)^2 - (\mathbf{D}\cdot\mathbf{D})(\mathbf{U}\cdot\mathbf{U} - r^2)}}{\mathbf{D}\cdot\mathbf{D}}.
\]
Evidently, the ray intersects our sphere when the discriminant of this equation is greater than or equal to $0$, i.e.
\[
	\big(\mathbf{D}\cdot\mathbf{U}\big)^2 - (\mathbf{D}\cdot\mathbf{D})(\mathbf{U}\cdot\mathbf{U} - r^2) \geq 0
\]
If this condition is met, the value of $t$ corresponding to the first intersection is the smallest value that is still greater than $0$.

\section*{Importance Sampling}

For a function $f(x)$ of a random variable $X$, its expected value is given by
\[
	\mathbb{E}_p[f(X)] = \int_{\Omega}f(x)p_X(x)\,\mathrm{d}x
\]
where $p_X(x)$ is the probability density function (PDF) of $X$. A statistical way to evaluate the expected value is given by the sample mean, defined by
\[
	\mathbb{E}_p[f(X)] = \lim_{N\to\infty}\frac{1}{N}\sum_{i=1}^Nf(x_i)
\]
where each $x_i$ is a realization of $X$ drawn from the distribution $p_X(x)$. We often want to evaluate integrals of the form $\int_\Omega f(x)\,\mathrm{d}x$, which we can do by taking the expected value of $f(x)/p_X(x)$ instead of just $f(x)$. Truncating the sample mean at a finite $N$, we see the correspondence
\[
	\int_\Omega f(x)\,\mathrm{d}x \approx \frac{1}{N}\sum_{i=1}^N\frac{f(x_i)}{p_X(x_i)}
\]
If the samples we use are not a good fit to $f$, we run the risk of adding up a lot of terms that contribute very little to the final result. We can work around this by including a simple multiplicative factor of $1$ in the expected value in the form of $q_X(x)/q_X(x)$, where $q_X(x)$ is a PDF that better matches $f$. Specifically, we see
\[
	\mathbb{E}_p\Big[\frac{f(X)}{p_X(X)}\frac{q_X(X)}{q_X(X)}\Big] = \int_\Omega \frac{f(x)}{p_X(x)}\frac{q_X(x)}{q_X(x)}p_X(x)\,\mathrm{d}x = \int_\Omega \frac{f(x)}{q_X(x)}q_X(x)\,\mathrm{d}x = \mathbb{E}_q\Big[\frac{f(X)}{q_X(X)}\Big]
\]
which gives us the correspondence
\[
\int_\Omega f(x)\,\mathrm{d}x \approx \frac{1}{N}\sum_{i=1}^N\frac{f(x_i)}{q_X(x_i)}
\]
where, in this case, each $x_i$ is sampled from the distribution defined by $q_X(x)$. There are, of course, restrictions on the possible choices of $q_X(x)$, but they are surprisingly soft: we simply require that $q_X(x)$ be nonzero wherever $f(x)p_X(x)$ is nonzero.

In path tracing, we use this to estimate the indirect lighting present in the scene. Specifically, we take $f(x)$ to be the integrand of the rendering equation and $N$ to be $1$,
\[
	\int_\Omega f_r(\mathbf{x}, \hat{\omega}_i, \hat{\omega}_o, \lambda, t)L_i(\mathbf{x}, \hat{\omega}_i, \lambda, t)|\hat{\omega}_i\cdot\mathbf{n}|\,\mathrm{d}\hat{\omega}_i \approx \frac{f_r(\mathbf{x}, \hat{\omega}_i, \hat{\omega}_o, \lambda, t)L_i(\mathbf{x}, \hat{\omega}_i, \lambda, t)|\hat{\omega}_i\cdot\mathbf{n}|}{p_\omega(\hat{\omega}_i)}
\]

\section*{Lambertian Reflectance}

Recall the rendering equation,

\[
L_o(\mathbf{x}, \hat{\omega}_o, \lambda, t) = L_e(\mathbf{x}, \hat{\omega}_o, \lambda, t) + \int_\Omega f_r(\mathbf{x}, \hat{\omega}_i, \hat{\omega}_o, \lambda, t)L_i(\mathbf{x}, \hat{\omega}_i, \lambda, t)|\hat{\omega}_i\cdot\mathbf{n}|\,\mathrm{d}\hat{\omega}_i
\]

where $f_r(\mathbf{x}, \hat{\omega}_i, \hat{\omega}_o, \lambda, t)$ describes the bidirectional reflectance distribution function (BRDF) of the material under consideration and $\Omega$ is the hemisphere centered at the geometry's normal vector.

A diffuse, Lambertian material has an equal chance to scatter light in all directions. This implies that $f_r(\mathbf{x}, \hat{\omega}_i, \hat{\omega}_o, \lambda, t) = \rho_0$ where $\rho_0$ is a constant. Because of this, the direction of the incoming radiance does not matter and we may take $L_i(\mathbf{x}, \hat{\omega}_i, \lambda, t)$ outside of the integral. Assuming all light is reflected imposes the additional requirement that
\[
	\int_\Omega \rho_0|\hat{\omega}_i\cdot\mathbf{n}|\,\mathrm{d}\hat{\omega}_i = \rho_0\int_\Omega\cos\theta\,\mathrm{d}\hat{\omega}_i = 1
\]
where $\theta$ is the angle between $\hat{\omega}_i$ and the normal. On the unit hemisphere, with $\phi$ describing the azimuthal angle and $\theta$ the altitude, this requirement becomes
\begin{align*}
	\rho_o\int_0^{2\pi}\int_0^{\pi/2}\cos\theta\sin\theta\,\mathrm{d}\theta\,\mathrm{d}\phi &= \frac{\rho_0}{2}\int_0^{2\pi}\int_0^{\pi/2}\sin(2\theta)\,\mathrm{d}\theta\,\mathrm{d}\phi \\
	&= \frac{\rho_0}{4}\int_0^{2\pi}{-\cos(2\theta)}\Big|_0^{\pi/2}\,\mathrm{d}\phi \\
	&= \frac{\rho_0}{2}\int_0^{2\pi}\mathrm{d}\phi \\
	&= \rho_0\pi \\
	&= 1
\end{align*}
	i.e. max reflectance occurs when $\rho_0 = 1/\pi$. If we allow this to vary by wavelength via $f_r(\mathbf{x}, \hat{\omega}_i, \hat{\omega}_o, \lambda, t) = \rho(\lambda)/\pi$ with $\rho(\lambda) \in [0, 1]$, the indirect part of the path tracing equation becomes
	\[
		\frac{\rho(\lambda)L_i(\mathbf{x}, \hat{\omega}_i, \lambda, t)|\hat{\omega}_i\cdot\mathbf{n}|}{\pi p_\omega(\hat{\omega}_i)}
	\]
	If we sample directions uniformly in the unit hemisphere, for example, $p_\omega(\hat{\omega}_i) = 1/2\pi$ and the above simplifies to
	\[
	2\rho(\lambda)L_i(\mathbf{x}, \hat{\omega}_i, \lambda, t)|\hat{\omega}_i\cdot\mathbf{n}|
	\]

\section*{Uniformly Sampling the Unit Sphere (and Unit Hemisphere)}

The easiest way to generate uniformly spaced points on the unit sphere is to build a cumulative distribution function (CDF), then use inverse transform sampling. To put it another way, suppose we want to generate samples from a PDF $p_X(x) = \mathrm{d}F_X(x)/\mathrm{d}x$ but only have access to $U$, a random variable uniformly distributed between $0$ and $1$. An obvious way to map from this domain to the range of $X$ is to define $Y = F_X^{-1}(U)$. The CDF of this new random variable is
\[
	F_Y(y) = \mathrm{Pr}(Y \leq y) = \mathrm{Pr}(F_X^{-1}(U) \leq y) = \mathrm{Pr}(U \leq F_X(y)) = F_X(y),
\]
where we have used the fact that all CDFs are monotonic and that $\mathrm{Pr}(U \leq u) = u$ for a random variable uniformly distributed between $0$ and $1$. Given the uniqueness of the derivative, we can be confident that $Y$ has the same distribution as $X$.

Moving on to the problem at hand, the PDF of the unit sphere with regard to solid angle $\Omega$ must be
\[
	p_\Omega(\Omega) = \frac{1}{4\pi}
\]
We can, of course, write this in $\phi$-$\theta$ space as
\[
	p_\Omega(\Omega)\,\mathrm{d}\Omega = p_{\theta,\phi}(\theta, \phi)\,\mathrm{d}\theta\mathrm{d}\phi
\]
where, upon identifying $\mathrm{d}\Omega = \sin\theta\,\mathrm{d}\theta\,\mathrm{d}\phi$, we find
\[
	p_{\theta,\phi}(\theta, \phi) = \frac{\sin\theta}{4\pi}
\]
We can integrate over each variable to obtain the PDF of the other,
\begin{align*}
	p_\theta(\theta) &= \int_0^{2\pi}\frac{\sin\theta}{4\pi}\mathrm{d}\phi = \frac{\sin\theta}{2} \\
	p_\phi(\phi) &= \int_0^{\pi}\frac{\sin\theta}{4\pi}\mathrm{d}\theta = \frac{1}{2\pi}
\end{align*}
which can then be used to find their respective CDFs
\begin{align*}
	F_\theta(\theta) &= \int_0^{\theta}\frac{\sin\theta'}{2}\mathrm{d}\theta' = \frac{1 - \cos\theta}{2} \\
	F_\phi(\phi) &= \int_0^{\phi}\frac{\mathrm{d}\phi'}{2\pi} = \frac{\phi}{2\pi}
\end{align*}
whose inverse functions are
\begin{align*}
	F_\theta^{-1}(u) &= \cos^{-1}(1 - 2u) \equiv \theta \\
	F_\phi^{-1}(v) &= 2\pi{v} \equiv \phi
\end{align*}
We can substitute these into the conversions from Cartesian to spherical coordinates to get
\begin{align*}
	x &= \sin\theta\cos\phi = \sqrt{1 - z^2}\cos(2\pi v) \\
	y &= \sin\theta\sin\phi = \sqrt{1 - z^2}\sin(2\pi v) \\
	z &= \cos\theta = 1 - 2u
\end{align*}

Another method for uniformly sampling the unit sphere is to draw random values from the unit normal distribution for each component of a vector. One then simply needs to normalize the resulting vector. This works because the combined Gaussian PDF is rotationally invariant, and thus must correspond to a uniform distribution on the sphere.

Yet another method is the so-called `rejection' method. In this, we draw values for the vector's components from a random variable uniformly distributed between $-1$ and $1$. If the vector lies within the unit ball, we normalize it. Otherwise, we repeat the process.

After we have setup a method to sample the unit sphere, sampling the unit hemisphere is just as easy. We simply dot the sampled vector with the desired normal vector: if this result is less than $0$, we negate the sampled vector.

\section*{Uniformly Sampling the Unit Disc}

We follow a similar strategy to above. Over the unit disc, a uniform sampling strategy must yield
\[
	p_A(A) = \frac{1}{\pi}.
\]
Identifying $\mathrm{d}A = r\,\mathrm{d}r\,\mathrm{d}\phi$ and equating the above PDF with the equivalent PDF in $r$-$\phi$ spaces gives
\[
	p_A(A)\,\mathrm{d}A = \frac{r}{\pi}\mathrm{d}r\,\mathrm{d}\phi = p_{r,\phi}(r, \phi)\,\mathrm{d}r\,\mathrm{d}\phi,
\]
i.e. $p_{r,\phi}(r,\phi) = r/\pi$. The PDF of each variable is
\begin{align*}
	p_r(r) &= \int_0^{2\pi}\frac{r}{\pi}\mathrm{d}\phi = 2r \\
	p_\phi(\phi) &= \int_0^1\frac{r}{\pi}\mathrm{d}r = \frac{1}{2\pi}
\end{align*}
and so the corresponding CDFs are
\begin{align*}
	F_r(r) &= \int_0^r 2r'\,\mathrm{d}r' = r^2 \\
	F_\phi(\phi) &= \int_0^{\phi}\frac{\mathrm{d}\phi'}{2\pi} = \frac{\phi}{2\pi}
\end{align*}
and so the inverse CDFs are
\begin{align*}
	F_r^{-1}(u) &= \sqrt{u} \equiv r \\
	F_\phi^{-1}(v) &= 2\pi{v} \equiv \phi
\end{align*}
In Cartesian coordinates, then, a randomly sampled point on the unit disc is given by
\begin{align*}
	x &= r\cos\phi = \sqrt{u}\cos(2\pi v) \\
	y &= r\sin\phi = \sqrt{u}\sin(2\pi v)
\end{align*}

\section*{Cosine Weighted Sampling of the Unit Hemisphere}

Looking back to the section on importance sampling, we see that sampling directions from a cosine weighted distribution has the advantage of removing the $|\hat{\omega}_i\cdot\mathbf{n}|$ term from the rendering equation. Let's figure out how to do this.

A cosine weighted distribution over the unit hemisphere should obey
\[
	p_{\Omega}(\Omega)\,\mathrm{d}\Omega = A\cos\theta\,\mathrm{d}\Omega = A\cos\theta\sin\theta\,\mathrm{d}\theta\,\mathrm{d}\phi
\]
where $A$ is a normalization constant. We can find it by integrating over the unit hemisphere,
\begin{align*}
	\int_{\Omega}p_{\Omega}(\Omega)\,\mathrm{d}\Omega &= A\int_0^{2\pi}\int_0^{\pi/2}\cos\theta\sin\theta\,\mathrm{d}\theta\,\mathrm{d}\phi \\
	&= \frac{A}{2}\int_0^{2\pi}\int_0^{\pi/2}\sin(2\theta)\,\mathrm{d}\theta\,\mathrm{d}\phi \\
	&= {-\frac{A}{2}}\int_0^{2\pi}\frac{\cos(2\theta)}{2}\Big|_0^{\pi/2}\mathrm{d}\phi \\
	&= \frac{A}{2}\int_0^{2\pi}\mathrm{d}\phi \\
	&= A\pi \\
	&= 1
\end{align*}
i.e. $A = 1/\pi$. Using this, we may rewrite the PDF as
\[
	p_{\theta,\phi}(\theta,\phi) = \frac{\cos\theta\sin\theta}{\pi} = \frac{\sin(2\theta)}{2\pi}
\]
The individual PDFs are
\begin{align*}
	p_{\theta}(\theta) &= \int_0^{2\pi}\frac{\sin(2\theta)}{2\pi}\mathrm{d}\phi = \sin(2\theta) \\
	p_{\phi}(\phi) &= \int_0^{\pi/2}\frac{\sin(2\theta)}{2\pi}\mathrm{d}\theta = -\frac{\cos(2\theta)}{4\pi}\Big|_0^{\pi/2} = \frac{1}{2\pi}
\end{align*}
and their respective CDFs are
\begin{align*}
	F_\theta(\theta) &= \int_0^\theta\sin(2\theta')\,\mathrm{d}\theta' = \frac{1 - \cos(2\theta)}{2} \\
	F_\phi(\phi) &= \int_0^\phi\frac{\mathrm{d}\phi'}{2\pi} = \frac{\phi}{2\pi}
\end{align*}
The inverse CDFs are
\begin{align*}
	F_\theta^{-1}(u) &= \frac{\cos^{-1}(1 - 2u)}{2} \equiv \theta \\
	F_\phi^{-1}(v) &= 2\pi{v} \equiv \phi
\end{align*}
Before expressing this in Cartesian coordinates, it is useful have the half-angle identities on hand
\[
	\cos\Big(\frac{x}{2}\Big) = \pm\sqrt{\frac{1 + \cos x}{2}} \qquad \sin\Big(\frac{x}{2}\Big) = \pm\sqrt{\frac{1 - \cos x}{2}}
\]
Using the above, we see
\[
	\cos\theta = \sqrt{1 - u} \qquad \sin\theta = \sqrt{u}
\]
where we have chosen the positive square root in each case so as to agree $\cos\theta$ and $\sin\theta$ when $\theta \in [0, \tfrac{\pi}{2}]$. Altogether this gives us
\begin{align*}
	x &= \sin\theta\cos\phi = \sqrt{u}\cos(2\pi v) \\
	y &= \sin\theta\sin\phi = \sqrt{u}\sin(2\pi v) \\
	z &= \cos\theta = \sqrt{1 - u}
\end{align*}
Comparing this to our strategy for randomly sampling the unit disc reveals that this is simply a projection of the former onto the unit hemisphere! Using this sampling method gives a particularly nice form to the integrand in the rendering equation in the case of a Lambertian surface,
\[
	\rho(\lambda)L_i(\mathbf{x}, \hat{\omega}_i, \lambda, t)
\]

\section*{Lack of Cosine Term in Specular Reflection}

Specular reflection requires that the integral term in the rendering equation,
\[
	\int_\Omega f_r(\mathbf{x}, \hat{\omega}_i, \hat{\omega}_o, \lambda, t)L_i(\mathbf{x}, \hat{\omega}_i, \lambda, t)|\hat{\omega}_i\cdot\mathbf{n}|\,\mathrm{d}\hat{\omega}_i
\]
conserve energy. In this particular case, the we know that $f_r(\mathbf{x}, \hat{\omega}_i, \hat{\omega}_o, \lambda, t) \propto \delta(\mathbf{n} - (\hat{\omega}_i + \hat{\omega_o})/2)$, and so, to conserve energy, we must have
\[
	\int_\Omega A\delta(\mathbf{n} - (\hat{\omega}_i + \hat{\omega_o})/2)L_i(\mathbf{x}, \hat{\omega}_i, \lambda, t)|\hat{\omega}_i\cdot\mathbf{n}|\,\mathrm{d}\hat{\omega}_i = L_i(\mathbf{x}, 2\mathbf{n} - \hat{\omega}_o, \lambda, t)
\]
where $A$ is a normalization constant. We see that the distribution must be
\[
	f_r(\mathbf{x}, \hat{\omega}_i, \hat{\omega}_o, \lambda, t) = \frac{\delta(\mathbf{n} - (\hat{\omega}_i + \hat{\omega_o})/2)}{|\hat{\omega}_i\cdot\mathbf{n}|}
\]
and hence $|\hat{\omega}_i\cdot\mathbf{n}|$ does not appear in implementations of specular reflection (or indeed any delta function distribution).

\section*{Russian Roulette}

Suppose we have an estimator $F$ that we replace with
an estimator $F'$
\[
	F' = \begin{cases}
		\frac{F}{p} & u \leq p \\
		0 & u > p
	\end{cases}
\]
where $u$ is a uniform random value between $0$ and $1$ and $p$ is some valid probability. In  words, $F'$ returns $F/p$ with chance $p$ and $0$ with chance $(1 - p)$. This new estimator has the same expected value as $F$, because
\[
	\mathbb{E}[F'] = p\cdot\frac{\mathbb{E}[F]}{p} + (1 - p)\cdot0 = \mathbb{E}[F]
\]

If we apply this to the indirect lighting integrand in our path tracer, we see that it allows the termination of a loop that would otherwise need to continue until hitting a light source---a process that could easily have no end. This has the advantage of allowing us to run our program in a finite time without incurring bias.

\section*{Reflection and Refraction}



\end{document}
